%----------------------------------------------------------------------------------------
% Tecnologico de Costa Rica
% Plantilla para cursos
%----------------------------------------------------------------------------------------

%----------------------------------------------------------------------------------------
%	PACKAGES AND OTHER DOCUMENT CONFIGURATIONS
%----------------------------------------------------------------------------------------
% tipo el documento
\documentclass[11pt,fleqn]{report} 

% Márgenes
\usepackage[top=3.5cm,bottom=3cm,left=5.5cm,right=3cm,headsep=20pt,letterpaper]{geometry}

% para uso del español
\usepackage[spanish,es-lcroman,es-nosectiondot]{babel}
% 8-bit encoding (para uso de glifos
\usepackage[T1]{fontenc} 

% Soporte para utf8
\usepackage[utf8]{inputenc}      

% Permite usar colores
%\usepackage{color}

% Permite definir colores
\usepackage{xcolor} 

% Azul TEC -> RGB:00447b
\definecolor{tec}{HTML}{00447B}
\definecolor{tecdark}{HTML}{003366}
\definecolor{gray8}{HTML}{333333}

\providecommand{\tightlist}{%
	  \setlength{\itemsep}{0pt}\setlength{\parskip}{0pt}}

% Para espaciado doble, 1.5 o simple
\usepackage{setspace}

%% Fuentes 
%Fuente GWC Goticq (ocupa el paquete fonts-urw-base35)
\usepackage{helvet} 
%\renewcommand*\familydefault{\sfdefault}


\usepackage{mathptmx} 

% Modificaciones para espaciado
\usepackage{microtype} 

%----------------------------------------------------------------------------------------

\usepackage[explicit]{titlesec}

% Para el uso de imágenes
\usepackage{graphicx}

% Para cuadros con cambio de linea
\usepackage{pbox}

% Para colocar cuadros flotantes
\usepackage{wrapfig}

%para las tablas grandes
\usepackage{longtable}

% Tablas que se autoajustan
\usepackage{tabularx}

% Este no se que putas hace
\usepackage{booktabs}

% Cambiar la fuente por defecto
\renewcommand{\familydefault}{\sfdefault}


% Pié de página

\usepackage{fancyhdr}
\usepackage{lastpage}
\usepackage{etoolbox}

\makeatletter
\def\footrule{{\if@fancyplain\let\footrulewidth\plainfootrulewidth\fi
    \vskip-\footruleskip\vskip-\footrulewidth
    \color{\footrulecolor}
    \leavevmode\rlap{\hspace*{-8em}\rule{16cm}{\footrulewidth}}
    \vskip\footruleskip
}}
\makeatother

\newcommand{\footrulecolor}{tec}
\renewcommand{\headrulewidth}{0pt}
\renewcommand{\footrulewidth}{0.5pt}

\pagestyle{fancy}
\fancyfoot{}
\fancyhf{}


\lhead{
    \begin{picture}(0,0)
        \put(220,0){\includegraphics[scale=0.45]{Pictures/logo_tec.png}}
    \end{picture}
}


\lfoot{
	\parindent=-8em
	\footnotesize 
    \textcolor{tec}{Escuela de Computación - Carrera de Ingeniería de Computación, Plan 410.}
}
\rfoot{
	\footnotesize 
    \textcolor{tec}{Página | \thepage \  de \pageref{LastPage} }
}

\title{Programa de Arquitectura de Computadoras}

\usepackage[pdftex, hidelinks]{hyperref}

\hypersetup{pdftitle={}}

%----------------------------------------------------------------------------------------
%	Creación de nuevos comandos
%----------------------------------------------------------------------------------------

% Cambia el texto del nombre del comando \chapter
\renewcommand{\chaptername}{Parte}

%\renewcommand{\thechapter}{\arabic{chapter}}

\newcounter{lastSectionCounter}
% Cambia el formato del nombre del comando \chapter
\renewcommand{\chapter}[1]{
  \newpage
  \parindent=-8em
  \setcounter{lastSectionCounter}{\value{section}}
  \stepcounter{chapter}
  \setcounter{section}{\value{lastSectionCounter}} 
  \textcolor{tecdark}{
        {
            \Large 
            \textbf{\thechapter \ Parte. #1}}}
  \parindent=0em
  \vspace{5mm}
  }

\titleformat{\section}[leftmargin]
    {
    %\titlerule*[.6em]{\bfseries.}%
    %\vspace{6pt}%
    \large
    \bfseries
    }
    {
        \color{tec} \thesection
    }{.5em}{\textcolor{tec}{#1}}

\titlespacing{\section}
{6pc}{1.5ex plus .1ex minus .2ex}{2pc}


\titleformat{\subsection}[hang]
    {
    %\color{red} 
    \bfseries
    }
    {\thesubsection}
    {0pc}{#1}

\titlespacing{\subsection}
{0pc}{0pt}{5pt}


% Cambia el formato de la sección
\renewcommand{\thesection}{\arabic{section}}
\renewcommand{\thechapter}{\Roman{chapter}}
%\setlength{\parskip}{0.5em}

% elimina la numeración de las subsecciones
\renewcommand\thesubsection{}

\setlength{\parskip}{10pt}


%----------------------------------------------------------------------------------------
%   BEGIN DOCUMENT  
%----------------------------------------------------------------------------------------

\begin{document}

%\makeatletter
%\def\footrule{{\if@fancyplain\let\footrulewidth\plainfootrulewidth\fi
%    \vskip-\footruleskip\vskip-\footrulewidth
%    \leavevmode\rlap{\hspace*{-8em}\rule{16cm}{\footrulewidth}}
%    \vskip\footruleskip}}
%\makeatother

\normalsize
\spanishsignitems

%----------------------------------------------------------------------------------------
%   Página de título	
%----------------------------------------------------------------------------------------
\thispagestyle{empty}

\begin{picture}(0,0)
    \put(-175,0){\includegraphics[scale=0.45]{Pictures/header_frontpage.png}}
\end{picture}


\parindent=-7em

~\vfill

\Large

\textbf{
Programa del curso
IC-3101 % Codigo Curso
}

\vspace{3mm}

\huge 

\begingroup
\hyphenpenalty 10000
\exhyphenpenalty 10000

\textbf{
		Arquitectura de Computadoras % NombreCurso
	}

\endgroup

\vspace{1cm}

\large 

\parindent=-4em

\textcolor{gray8}{
    \textbf{
            Escuela de Computación % Escuela
        }
}

\textcolor{gray8}{
    \textbf{
            Carrera de Ingeniería de Computación, Plan 410. % Plan
        }
}

\normalsize 

\parindent=0em



%----------------------------------------------------------------------------------------
%	Parte I: Pagina de Plan de Estudios del curso
%----------------------------------------------------------------------------------------

\chapter{Aspectos relativos al plan de estudios}


\parindent=-7em

\textcolor{tec}{
  {\large \textbf{1 Datos generales}}}

\parindent=-6em

\doublespacing
  \begin{tabularx}{\textwidth}{p{6cm}p{10cm}}
        \textbf{Nombre del curso:}                  & Arquitectura de Computadoras         \\
        \textbf{Código:}                            & IC-3101         \\
        \textbf{Tipo de curso:}                     & Teórico - Práctico           \\
                                                    &                       \\
        \textbf{Nº de créditos:}                    & 4            \\
        \textbf{Nº horas de clase por semana:}      & 4          \\
        \textbf{Nº horas extraclase por semana:}    & 8     \\
                                                    &                       \\
        \textbf{Ubicación en el plan de estudios:}  & Curso del 2do semestre de la carrera de Ingeniería en Computación       \\
        \textbf{Requisitos:}                        & IC1400 Fundamentos de Organización de Computadoras.          \\
        \textbf{Correquisitos:}                     & IC-2001 Estructuras de Datos       \\ 
        \textbf{El curso es requisito de:}          & Ninguno         \\
        \textbf{Asistencia:}                        & Obligatoria          \\
        \textbf{Suficiencia:}                       & No         \\ 
        \textbf{Posibilidad de reconocimiento:}     & No      \\
        \textbf{Vigencia del programa:}             & II semestre 2015.            \\
    \end{tabularx}
\singlespacing


\parindent=0em

%----------------------------------------------------------------------------------------
%	Cuerpo del programa de Curso
%----------------------------------------------------------------------------------------

\setcounter{section}{1}

\newpage
\section{Descripción General}\label{descripciuxf3n-general}

En este curso se estudiarán todos aquellos recursos de la computadora
que son visibles al programador de bajo nivel. El estudiante será
preparado en programación en lenguaje ensamblador para una arquitectura
particular, aunque al mismo tiempo se estudian conceptos arquitectónicos
presentes de manera extendida en computadoras actuales.

\section{Objetivos}\label{objetivos}

\subsection{Objetivo General}\label{objetivo-general}

Analizar el conocimiento sobre una arquitectura de computadoras actual
para la programación desde una perspectiva de bajo nivel.

\subsection{Objetivos Específicos}\label{objetivos-especuxedficos}

\begin{itemize}
\tightlist
\item
  Programar de manera competente en el lenguaje ensamblador de al menos
  una arquitectura.
\item
  Entender los alcances y limitaciones que una arquitectura establece a
  un potencial programador.
\item
  Dominar conceptos universales de Arquitectura de Computadoras
  presentes en diversas computadoras actuales.
\item
  Comprender el funcionamiento general de sistemas empotrados.
\end{itemize}

\section{Contenidos}\label{contenidos}

\subsection{Introducción (1 semana)}\label{introducciuxf3n-1-semana}

\begin{itemize}
\tightlist
\item
  Organización vs.~Arquitectura de Computadoras
\item
  Conceptos generales y Repaso histórico
\item
  Medidas de rendimiento
\item
  ¿Por qué aprender ensamblador?
\item
  Arquitectura ejemplo
\end{itemize}

\subsection{Infraestructura de software (0.5
semanas)}\label{infraestructura-de-software-0.5-semanas}

\begin{itemize}
\tightlist
\item
  Ensamblador
\item
  Linker
\item
  Loader
\item
  Lenguajes de Alto nivel
\item
  Simuladores
\item
  Relación con Sistema Operativo
\end{itemize}

\subsection{Aritmética para computadoras (2
semanas)}\label{aritmuxe9tica-para-computadoras-2-semanas}

\begin{itemize}
\tightlist
\item
  Repaso Sistemas Numéricos
\item
  Diseño general de una ALU
\item
  Enteros con y sin signo
\item
  Suma, resta y operaciones lógicas
\item
  Multiplicación y División
\item
  Punto Flotante
\item
  Aproximaciones
\end{itemize}

\subsection{Lenguaje Ensamblador (6
semanas)}\label{lenguaje-ensamblador-6-semanas}

\begin{itemize}
\tightlist
\item
  Conceptos introductorios.
\item
  Formatos de Instrucción.
\item
  Modos de Direccionamiento.
\item
  Conjunto de instrucciones básico.
\item
  Proceso de ensamblaje - desensamblaje.
\item
  Programación básica con el ensamblador (aritmética y flujo de
  control).
\item
  Paso de parámetros y construcción de rutinas.
\item
  Macros y preensamblaje.
\item
  Directivas y Operadores avanzados.
\item
  Modularización, ``Linking'' y ``Loading''
\item
  Formatos de módulos ejecutables
\item
  Comunicación entre Programas
\end{itemize}

\subsection{Pipeline y modelos de alto rendimiento (2
semanas)}\label{pipeline-y-modelos-de-alto-rendimiento-2-semanas}

\begin{itemize}
\tightlist
\item
  Introducción y conceptos básicos
\item
  Paralelismo vs concurrencia
\item
  Pipeline conceptos y dificultades de implantación
\item
  Pipeline de instrucciones
\item
  Predicción de bifurcaciones
\item
  Otros modelos de alto rendimiento
\end{itemize}

\subsection{RISC vs CISC (0.5 semanas)}\label{risc-vs-cisc-0.5-semanas}

\begin{itemize}
\tightlist
\item
  Diferencias y características de ejecución
\item
  Ventajas de CISC y Ventajas de RISC
\end{itemize}

\subsection{Entrada/Salida (1 semana)}\label{entradasalida-1-semana}

\begin{itemize}
\tightlist
\item
  Conceptos básicos
\item
  Tipos y características de dispositivos de E/S
\item
  Interfaces E/S con memoria, CPU y Sistema Operativo
\item
  DMA -- E/S mapeada a memoria
\item
  Diseño de Sistemas de E/S
\end{itemize}

\subsection{Multiprocesadores (2
semanas)}\label{multiprocesadores-2-semanas}

\begin{itemize}
\tightlist
\item
  Introducción
\item
  Conceptos de Programación
\item
  Organizaciones
\item
  Interconection Network
\item
  Sistemas Multicore
\item
  Threads de hardware
\item
  Multiprocesadores y jerarquía de memorias
\item
  Protocolos de coherencia
\end{itemize}

\subsection{Sistemas empotrados (1
semana)}\label{sistemas-empotrados-1-semana}

\begin{itemize}
\tightlist
\item
  Definiciones
\item
  System-on-chip
\item
  Aplicaciones
\end{itemize}

\chapter{Aspectos operativos}\label{aspectos-operativos}

\section{Metodología de enseñanza y
aprendizaje}\label{metodologuxeda-de-enseuxf1anza-y-aprendizaje}

Clases magistrales donde el profesor desarrollará el material teórico
asociado al curso. Se seguirán dos grandes ramas: entrenamiento en
programación ensamblador y conceptos de arquitectura de computadoras. Se
recomienda estos dos temas que no sean cubiertos de manera secuencial,
sino de manera intercalada para una mejor comprensión e ilustración de
los temas aprendidos en cada rama. Los estudiantes deberán estudiar
independientemente material adicional asignado en la forma de lecturas o
investigaciones bibliográficas, y desarrollarán una serie de proyectos
de programación en lenguaje ensamblador usando preferiblemente equipos
instalados en nuestros laboratorios o en su defecto recurriendo a
simuladores de software de las arquitecturas estudiadas.\\
En lo posible se contrastará el ensamblador ejemplo con algún otro
ensamblador que permita apreciar diferencias sustanciales entre ellos.

\section{Evaluación}\label{evaluaciuxf3n}

A continuación se detalla la evaluación del curso:

\begin{longtable}[]{@{}ll@{}}
\toprule
& \%\tabularnewline
\midrule
\endhead
Proyectos programados & 40\tabularnewline
Pruebas o tareas cortas & 25\tabularnewline
Exámenes & 15\tabularnewline
Laboratorios & 20\tabularnewline
& 100\tabularnewline
\bottomrule
\end{longtable}

El contenido académico de las actividades, llámense tareas o pruebas
cortas, son acumulativos. Las pruebas cortas se efectuarán en el momento
de la clase que el profesor considere más apropiado. El porcentaje de
las pruebas cortas y tareas será el promedio de las notas obtenidas en
cada una de ellas. Las pruebas cortas no se reponen, se debe llegar a
tiempo a las presentaciones.

En caso de que se detecte un plagio o intento de fraude en cualquier
asignación por parte de un estudiante se procederá según la
reglamentación del Tecnológico. No se aceptarán trabajos 10 minutos
después de la fecha y hora indicadas. Por lo tanto, trabajos entregados
tardíamente tendrán una nota de cero.

\textbf{Aspectos Administrativos:}

\begin{itemize}
\tightlist
\item
  El curso se aprueba con nota de 70. No hay examen de reposición.
\item
  El profesor se guarda el derecho de \textbf{revisar la ortografía},
  redacción y coherencia con puntos negativos, en las asignaciones.
\item
  Todas las asignaciones escritas deben de presentarse en formato
  \textbf{pdf}.
\item
  Se debe de realizar la entrega de todos los proyectos y exámenes
  asignados que posean un valor mayor o igual a un 10\%, de lo contrario
  se considerará un abandono de curso.
\item
  Las pruebas cortas pueden consistir en una asignación para trabajo
  fuera de clase.
\item
  Queda agendado un \textbf{quiz} en cada una de las lecciones del
  curso, el profesor tomará la decisión de aplicar o no el quiz.
\item
  \textbf{No es permitido grabar fotografías}, videos o audio durante la
  clase.
\end{itemize}

\section{Bibliografía}\label{bibliografuxeda}

John Henessy, David Patterson, ``Computer Architecture: A Quantitative
Approach'', 5ta. Edición, Morgan Kauffman, 2012.

Tanenbaum, Andrew, ``Structured Computer Organization'', 5ta. Edición,
Prentice Hall, 2005.

Duran, Luis. El gran libro del PC Interno. Grupo Marcombo ediciones
técnicas, primera edición, Barcelona 2007

Irving, Kip. Lenguaje Ensamblador para computadoras basadas en Intel.
Pearson Prentice Hall. Quinta edición. México, 2008

David Patterson, John Henessy, ``Computer Organization \& Design'', 4ta.
Edición, Morgan Kauffman, 2009.

Stallings, William. Organización y Arquitectura de Computadores. Pearson
Educación, setima edición, México 2006.

William Hohl, ``ARM Assembly Language'', CRC Press, 2009

Steve Furber, ``ARM System-on-Chip Architecture'', Pearon Education,
2000

Englander, Irv. Arquitectura computacional. Compañía editorial
continental, primera edición, México 2002

Thomas Rauber y Gudula Rünger, ``Parallel Programming: for Multicore and
Cluster Systems''. Editorial Springer-Verlag. Alemania 2010.

%----------------------------------------------------------------------------------------
%	Información del profesor
%----------------------------------------------------------------------------------------

\section{Profesor}

Ing. Jaime Gutiérrez Alfaro. \vspace{5mm} \\
\textbf{Email:} jaime.cr@gmail.com \\
\textbf{Página:} http://www.ic-itcr.ac.cr/\textasciitilde{}jgutierrez \\
\textbf{Twitter:} elotrojames
\\
\\
\textbf{Horario y lugar de consulta:} Jueves de 11:00 a 13:00 y Viernes de 17:00 a 19:00.. Oficinas Administrativas Alajuela.


\end{document}
