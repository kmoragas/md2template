\documentclass{article}
\usepackage[spanish]{babel}
% \usepackage{lipsum}
% \usepackage{natbib}
% \usepackage{graphicx}
\usepackage{analysis_orax}
\usepackage{changepage}
\usepackage{longtable}
\usepackage{booktabs}
\usepackage{enumitem}% \setlist

\usepackage{soul}

\usepackage{type1cm}
\usepackage{lettrine}

\usepackage[colorlinks=true,urlcolor=blue,linkcolor=black,anchorcolor=blue]{hyperref}

\usepackage[defaultlines=10,all]{nowidow}

\providecommand{\tightlist}{%
          \setlength{\itemsep}{0pt}\setlength{\parskip}{0pt}}
          
\newcommand{\headerline}[3]{%
  \par\medskip\noindent
  \makebox[0pt][l]{#1}%
  \makebox[20.5cm][c]{\textbf{#2}}%
  \makebox[0pt][r]{#3}\par\medskip}

\definecolor{bannergray}{HTML}{2E2F34}
\definecolor{bannerred}{HTML}{F07468}
\definecolor{bannertec}{HTML}{00447B}

\setlength{\footskip}{1.5cm}

\usepackage{datetime}

\providecommand{\tightlist}{%
          \setlength{\itemsep}{0pt}\setlength{\parskip}{0pt}}

\newdateformat{monthyeardate}{%
  \monthname[\THEMONTH], \THEYEAR}


%------------------Document----------------------------

\begin{document}

%-------------------------TitlePage------------------------
%\begin{titlepage}
%\end{titlepage}

\begingroup
\thispagestyle{empty}
\begin{tikzpicture}[remember picture,overlay]

\coordinate [below=5.15cm,left=-5cm] (midpoint) at (current page.north);
\coordinate [below=29.2cm,left=-5cm] (downpoint) at (current page.north);

\node at (current page.north west)
{\begin{tikzpicture}[remember picture,overlay]
\node[anchor=north west,inner sep=0pt] at (0,-2.3) {\includegraphics[width=\paperwidth]{figures/PIA21073large.jpg}}; % Background image

\node[anchor=north west,inner sep=0pt] at (1,-0.5cm) {\includegraphics[scale=0.4]{figures/logo_tec.png}}; % logotec

\makeatletter
\renewcommand{\monthnamespanish}[1][\month]{%
  \@orgargctr=#1\relax
  \ifcase\@orgargctr
    \PackageError{datetime}{Invalid Month number \the\@orgargctr}{%
      Month numbers should go from 1 to 12}%
    \or Enero%
    \or Febrero%
    \or Marzo%
    \or Abril%
    \or Mayo%
    \or Junio%
    \or Julio%
    \or Agosto%
    \or Septiembre%
    \or Octubre%
    \or Noviembre%
    \or Diciembre%
    \else \PackageError{datetime}{Invalid Month number \the\@orgargctr}{%
      Month numbers should go from 1 to 12}%
  \fi}
\makeatother

\draw[anchor=north] (midpoint) node [fill=bannergray,fill opacity=0.8,text opacity=1,inner sep=1cm]{

	\Huge\bfseries\sffamily\parbox[c][][t]{\paperwidth}{
			\centering \Large \color{white} $Asignacion$ \\[5pt] % Book title
			{\Huge \color{bannerred}  \fontspec{Verdana} \so {$Titulo$}	}\\[20pt] % Subtitle
			{\small \monthyeardate\today}
			}

}; % Proyect Name

\draw[anchor=north] (downpoint) node [fill=bannertec,fill opacity=1,text opacity=1,inner sep=0.5cm]{

	\color{white}\normalsize\sffamily\parbox[c][][t]{\paperwidth}{
%			\centering \color{white} $Escuela$ - $NombreCurso$ - Prof. $Profesor$
			\headerline{$Plan$}{$NombreCurso$}{Prof. $Profesor$}
		}

}; % Author


\end{tikzpicture}};
\end{tikzpicture}
\vfill
\endgroup

\newpage

\begin{tikzpicture}[remember picture,overlay]

\coordinate [below=10.15cm,left=2.8cm] (objectivopoint) at (current page.north);
\coordinate [below=10.5cm,left=2.8cm] (datosgeneralespoint) at (current page.north);

\draw[anchor=south] (objectivopoint) node [fill=bannertec,fill opacity=0.8,text opacity=1,inner sep=1cm]{

	\color{white}\sffamily\parbox[c][6cm][t]{6cm}{
			\vfill
			{\bfseries \huge  Objetivo} \\[5pt] % Book title
			\large $Objetivo$	
		}

}; % Objetivo

\draw[anchor=north] (datosgeneralespoint) node [fill=bannergray,fill opacity=0.8,text opacity=1,inner sep=1cm]{

	\color{white}\sffamily\parbox[c][18.5cm][t]{6cm}{
			{\bfseries \huge  Datos Generales} \\[5pt] % Book title
			\large 
			\begin{itemize}
				\item \textbf{Fecha de Entrega:} \\ $FechaEntrega$.
				\item \textbf{Fecha de Revisión:} \\ $FechaRevision$.
				\item \textbf{Lenguaje:} \\ $Lenguaje$
				\item \textbf{Recurso Humano:}\\ $RecursoHumano$
				\item \textbf{Valor de la asignación:} $Porcentaje$\%				
			\end{itemize}			
			\vspace{1cm}
			{\bfseries \huge  Profesor} \\[5pt] % Book title
			$Profesor$\\
			$Email$\\
			$Escuela$
			
		}

}; % Objetivo



\end{tikzpicture}



\begin{adjustwidth}{8.8cm}{}
~\vfill

%-------------------------Content------------------------
\section{Introducción}\label{introducciuxf3n}
\large
\begin{sloppypar}
$Introduccion$
\end{sloppypar}

\vspace{1.5cm}

\end{adjustwidth}


\newcommand{\sectionbreak}{\clearpage}

$body$

~\vfill

% %-------------------------Content------------------------


\end{document}
