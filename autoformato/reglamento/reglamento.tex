
\documentclass[10pt]{article}
\usepackage{graphicx,amssymb, amstext, amsmath, epstopdf, booktabs, verbatim, gensymb, geometry, appendix, natbib, lmodern}
\geometry{letterpaper}
%\usepackage{garamond}

\usepackage[colorlinks=true,urlcolor=blue,linkcolor=black,anchorcolor=blue]{hyperref}


$if(Lenguaje)$
\usepackage[spanish,es-lcroman,es-nosectiondot]{babel}
\usepackage[utf8]{inputenc}
$endif$

$if(Titulo)$
\newcommand*\Title{$Titulo$}
$else$
\newcommand*\Title{An Example Document}
$endif$

$if(Email)$
\newcommand*\Email{$Email$}
$else$
\newcommand*\Email{An Example Document}
$endif$

$if(Instituto)$
\newcommand*\Instituto{$Instituto$}
$else$
\newcommand*\Instituto{An Example Document}
$endif$


$if(Tema)$
\newcommand*\cpiType{$Tema$}
$else$
\newcommand*\cpiType{CPI Template}
$endif$

\newcommand*\Date{\today}


\newcommand*\Author{$Autor$}


$if(Ubicacion)$
\newcommand*\Location{$Ubicacion$}
$else$
\newcommand*\Location{TEC}
$endif$


$if(Titulo)$
\title{$Titulo$}
$else$
\title{An Example Document}
$endif$



\author{$Autor$}
\date{\today}
%-----------------------------------------------------------

\usepackage{cpistuff/cpi} % This is what makes your document look like a cpi document.


\begin{document}

$if(Titulo)$

\begin{center}
{\rmfamily \Large \color{cpiOrange} $Titulo$ \\ \vspace{12pt}}
\end{center}


$else$
An Example Document
$endif$

{\rmfamily \Large \color{cpiOrange} Tipo de Reglamento \\ \vspace{6pt}}



$TipoReglamento$

\tableofcontents

\pagebreak


{\rmfamily \Large \color{cpiOrange} Objetivo General \\ \vspace{6pt}}

$ObjetivoGeneral$

%\linespread{1.15} %Set standard document linespacing


$body$

{\vspace{50pt} \rmfamily \Large \color{cpiOrange} Fecha de entrada en vigencia \\ \vspace{6pt}}

$FechaVigencia$

\end{document}
              
