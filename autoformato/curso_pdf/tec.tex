%----------------------------------------------------------------------------------------
% Tecnologico de Costa Rica
% Plantilla para cursos
%----------------------------------------------------------------------------------------

%----------------------------------------------------------------------------------------
%	PACKAGES AND OTHER DOCUMENT CONFIGURATIONS
%----------------------------------------------------------------------------------------
% tipo el documento
\documentclass[11pt,fleqn]{report} 

% Márgenes
\usepackage[top=3.5cm,bottom=3cm,left=5.5cm,right=3cm,headsep=20pt,letterpaper]{geometry}

% para uso del español
\usepackage[spanish,es-lcroman,es-nosectiondot]{babel}
% 8-bit encoding (para uso de glifos
\usepackage[T1]{fontenc} 

% Soporte para utf8
\usepackage[utf8]{inputenc}      

% Permite usar colores
%\usepackage{color}

% Permite definir colores
\usepackage{xcolor} 

% Azul TEC -> RGB:00447b
\definecolor{tec}{HTML}{00447B}
\definecolor{tecdark}{HTML}{003366}
\definecolor{gray8}{HTML}{333333}

% Para espaciado doble, 1.5 o simple
\usepackage{setspace}

%% Fuentes 

%Fuente Avantgarde
%\usepackage{avant} 

%Fuente Helvet
\usepackage{helvet}

\usepackage{mathptmx} 

% Modificaciones para espaciado
\usepackage{microtype} 

%----------------------------------------------------------------------------------------

\usepackage[explicit]{titlesec}

% Para el uso de imágenes
\usepackage{graphicx}

% Para cuadros con cambio de linea
\usepackage{pbox}

% Para colocar cuadros flotantes
\usepackage{wrapfig}

%para las tablas grandes
\usepackage{longtable}

% Tablas que se autoajustan
\usepackage{tabularx}

% Este no se que putas hace
\usepackage{booktabs}

% Cambiar la fuente por defecto
\renewcommand{\familydefault}{\sfdefault}


% Pié de página

\usepackage{fancyhdr}
\usepackage{lastpage}
\usepackage{etoolbox}

\makeatletter
\def\footrule{{\if@fancyplain\let\footrulewidth\plainfootrulewidth\fi
    \vskip-\footruleskip\vskip-\footrulewidth
    \color{\footrulecolor}
    \leavevmode\rlap{\hspace*{-8em}\rule{16cm}{\footrulewidth}}
    \vskip\footruleskip
}}
\makeatother

\newcommand{\footrulecolor}{tec}
\renewcommand{\headrulewidth}{0pt}
\renewcommand{\footrulewidth}{0.5pt}

\pagestyle{fancy}
\fancyfoot{}
\fancyhf{}


\lhead{
    \begin{picture}(0,0)
        \put(220,0){\includegraphics[scale=0.45]{Pictures/logo_tec.png}}
    \end{picture}
}


\lfoot{
	\parindent=-8em
	\footnotesize 
    \textcolor{tec}{$Escuela$ - $Plan$}
}
\rfoot{
	\footnotesize 
    \textcolor{tec}{Página | \thepage \  de \pageref{LastPage} }
}

$if(NombreCurso)$
\title{Programa de $NombreCurso$}
$else$
\title{Programa de IC-XXXX} % Codigo Curso no conocido
$endif$

\usepackage[pdftex, hidelinks]{hyperref}

\hypersetup{pdftitle={$title-meta$}}

%----------------------------------------------------------------------------------------
%	Creación de nuevos comandos
%----------------------------------------------------------------------------------------

% Cambia el texto del nombre del comando \chapter
\renewcommand{\chaptername}{Parte}

%\renewcommand{\thechapter}{\arabic{chapter}}

\newcounter{lastSectionCounter}
% Cambia el formato del nombre del comando \chapter
\renewcommand{\chapter}[1]{
  \newpage
  \parindent=-8em
  \setcounter{lastSectionCounter}{\value{section}}
  \stepcounter{chapter}
  \setcounter{section}{\value{lastSectionCounter}} 
  \textcolor{tecdark}{
        {
            \Large 
            \textbf{\thechapter \ Parte. #1}}}
  \parindent=0em
  \vspace{5mm}
  }

\titleformat{\section}[leftmargin]
    {
    %\titlerule*[.6em]{\bfseries.}%
    %\vspace{6pt}%
    \large
    \bfseries
    }
    {
        \color{tec} \thesection
    }{.5em}{\textcolor{tec}{#1}}

\titlespacing{\section}
{6pc}{1.5ex plus .1ex minus .2ex}{2pc}


\titleformat{\subsection}[hang]
    {
    %\color{red} 
    \bfseries
    }
    {\thesubsection}
    {0pc}{#1}

\titlespacing{\subsection}
{0pc}{0pt}{5pt}


% Cambia el formato de la sección
\renewcommand{\thesection}{\arabic{section}}
\renewcommand{\thechapter}{\Roman{chapter}}
%\setlength{\parskip}{0.5em}

% elimina la numeración de las subsecciones
\renewcommand\thesubsection{}

\setlength{\parskip}{10pt}


%----------------------------------------------------------------------------------------
%   BEGIN DOCUMENT  
%----------------------------------------------------------------------------------------

\begin{document}

%\makeatletter
%\def\footrule{{\if@fancyplain\let\footrulewidth\plainfootrulewidth\fi
%    \vskip-\footruleskip\vskip-\footrulewidth
%    \leavevmode\rlap{\hspace*{-8em}\rule{16cm}{\footrulewidth}}
%    \vskip\footruleskip}}
%\makeatother

\normalsize
\spanishsignitems

%----------------------------------------------------------------------------------------
%   Página de título	
%----------------------------------------------------------------------------------------
\thispagestyle{empty}

\begin{picture}(0,0)
    \put(-175,0){\includegraphics[scale=0.45]{Pictures/header_frontpage.png}}
\end{picture}


\parindent=-7em

~\vfill

\Large

\textbf{
Programa del curso
$if(CodigoCurso)$
$CodigoCurso$ % Codigo Curso
$else$
IC-XXXX % Codigo Curso no conocido
$endif$
}

\vspace{3mm}

\huge 

\textbf{
	$if(NombreCurso)$
	$NombreCurso$ % NombreCurso
	$else$
	(Curso sin nombre) % NombreCurso
	$endif$
}

\vspace{1cm}

\large 

\parindent=-4em

\textcolor{gray8}{
    \textbf{
    $if(Escuela)$
        $Escuela$ % Escuela
    $else$
        (Escuela no especificada) % Escuela
    $endif$
    }
}

\textcolor{gray8}{
    \textbf{
    $if(Plan)$
        $Plan$ % Plan
    $else$
        (Plan no especificado) % Plan
    $endif$
    }
}

\normalsize 

\parindent=0em



%----------------------------------------------------------------------------------------
%	Parte I: Pagina de Plan de Estudios del curso
%----------------------------------------------------------------------------------------

\chapter{Aspectos relativos al plan de estudios}


\parindent=-7em

\textcolor{tec}{
  {\large \textbf{1 Datos generales}}}

\parindent=-6em

\doublespacing
  \begin{tabularx}{\textwidth}{p{6cm}p{10cm}}
        \textbf{Nombre del curso:}                  & $NombreCurso$         \\
        \textbf{Código:}                            & $CodigoCurso$         \\
        \textbf{Tipo de curso:}                     & $TipoCurso$           \\
                                                    &                       \\
        \textbf{Nº de créditos:}                    & $Creditos$            \\
        \textbf{Nº horas de clase por semana:}      & $HorasClase$          \\
        \textbf{Nº horas extraclase por semana:}    & $HorasExtraClase$     \\
                                                    &                       \\
        \textbf{Ubicación en el plan de estudios:}  & $UbicacionPlan$       \\
        \textbf{Requisitos:}                        & $Requisitos$          \\
        \textbf{Correquisitos:}                     & $Correquisitos$       \\ 
        \textbf{El curso es requisito de:}          & $RequisitoDe$         \\
        \textbf{Asistencia:}                        & $Asistencia$          \\
        \textbf{Suficiencia:}                       & $Suficiencia$         \\ 
        \textbf{Posibilidad de reconocimiento:}     & $Reconocimiento$      \\
        \textbf{Vigencia del programa:}             & $Vigencia$            \\
    \end{tabularx}
\singlespacing


\parindent=0em

%----------------------------------------------------------------------------------------
%	Cuerpo del programa de Curso
%----------------------------------------------------------------------------------------

\setcounter{section}{1}

\newpage
$body$

%----------------------------------------------------------------------------------------
%	Información del profesor
%----------------------------------------------------------------------------------------

\section{Profesor}

$Profesor$ \vspace{5mm} \\
\textbf{Email:} $Email$ \\
\textbf{Página:} $PaginaProfesor$ \\
\textbf{Twitter:} $Twitter$
\\
\\
\textbf{Horario y lugar de consulta:} $HorarioConsulta$

Los medios de comunicación oficiales del curso y contactos de los profesores serán brindados en la primera semana de clases.
\\
\\
$if(SinopsisProfesor)$
$SinopsisProfesor$ % NombreCurso
$endif$

\end{document}
